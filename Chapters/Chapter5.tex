% Chapter Template

\chapter{Gamma Ray Laser: Unresolved Problems} % Main chapter title

\label{Chapter5} % Change X to a consecutive number; for referencing this chapter elsewhere, use \ref{ChapterX}

%----------------------------------------------------------------------------------------
%	SECTION 1
%----------------------------------------------------------------------------------------

\section{Summary}

Although a design has been proposed which should in principle produce the desired $\gamma$-ray laser, there are a number of issues at all stages of the process that must be resolved before the design can be implemented in practice. These are given by the part of the process they pertain to.

%-----------------------------------
%	SUBSECTION 1
%-----------------------------------
\section{Positrons}

Of the positrons that will be produced most will be lost before they can produce Positronium. The positron beam produced has a width that is $1.3mm$ (approximately). In comparison the cavity itself will be about $0.1\mu m$. Assuming a silica target of the order of $1\mu m$, the vast majority of the positrons will never strike the target and thus will be unable to produce positrons. This is further compounded by the limitations on the number of positrons that can be trapped. At this point $10^7$ can be kept. To counter this there will need to be further work into the use of multiple or combined trapping mechanisms.
\newline
\newline
After reaching the target, most of the remaining positrons will either annihilate with an electron in the target, or will form para-Positronium (lifespan around $120 ns$), although as shown beforehand aerogel limits the losses to $60\%$.
\newline
\newline
The critical temperature for BEC formation is dependent on the density of the ortho-Positronium atoms to the $2/3$ power. Thus it is desirable to raise the density of the positrons so as to increase that of the Positronium and thus increase the critical temperature. If this can be achieved, then the time required for cooling the Positronium will be reduced and less will have decayed before a BEC can be formed and annihilation stimulated.
%-----------------------------------
%	SUBSECTION 2
%-----------------------------------

\section{Materials}
There remains a great uncertainty over the choice and use of a material which will provide not only the target for the positron beam but the means by which Positronium will be produced and the cavity where it will be cooled and annihilated. The key issues are:
\begin{itemize}
	\item   Pore density, size and distribution: The optimal configuration to prevent spin quenching is to have small, randomly distributed pores. If too many pores are produced however this will lead to an unequal distribution in the Positronium atoms and will not allow a sufficient density to be reached in many of the cavities, thus effectively losing more Positronium without further laser production.
	\item Pore shape: In a spherical pore, there is no preferred direction for the annihilation and thus $\gamma$-rays will be emitted equally in every direction. To limit this the intention is to produce pores which are longer, and perhaps narrower. This increases the Positronium along the desired axis and thus the laser will be more intense along this direction.
	\item Manufacturing process: Although a certain configuration of pores is desired, it is not yet known how this could be achieved.
	\item Paramagnetic defects: Unpaired electrons will be induced to oscillate by the UV cooling lasers, producing magnetic fields. These will affect the incoming positrons and the Positronium atoms moving through to the cavity, in a manner that will result in the loss of many. Thus it is desirable to find a material which is not subject to these.
	\item Target resilience: Bombardment by protons could cause substantial damage to the structure of the target. This will affect the rate at which Positronium can be successfully produced, so the useful lifetime of the target must be ascertained.
	\item Premature annihilation: Certain materials (e.g. carbon nanotubes) suffer from causing annihilation on their surfaces or within. The desired material must not be the cause of such a problem if it is to be of use.
\end{itemize}
%----------------------------------------------------------------------------------------
%	SECTION 2
%----------------------------------------------------------------------------------------

\section{Cooling}
Problems pertaining to this are closely related to the materials, and also affect the ability to form a BEC at a given density. It is desirable to find a method to cool the Positronium as far as possible.
\begin{itemize}
	\item Target Temperature: In order to suppress Ps2 formation (and the premature accompanying annihilation), the target must be kept hot, $300K$ may suffice but this must be ascertained. This will affect the rate at which cooling can occur by effectively eliminating the thermalisation process. The laser will itself be wholly dependent on laser cooling.
	\item Positronium lifetime: Even with thermalisation, Positronium will not survive for a sufficient time to be cooled to a BEC in its ground state. Using UV lasers the Positronium can be excited from the 1s to 2p states which can double or even triple the lifetime (Lifetime of 2p ~ $3.2ns$ but this ‘resets’ the lifetime of the 1s post decay. Excitations can occur more than once).
	\item Laser induced paramagnetism: As discussed before, a material must be found that will not be subject to this.
\end{itemize}

\section{Stimulated Annihilation}
To induce annihilation a low energy IR pulse will be sent which will change the spin state of the ortho-Positronium to para-Positronium. The wavelength will need to be known and apparatus assembled to produce said pulse. It is also not known how much of the Positronium will change spin states as a result. The duration must be short enough so as not to emerge (and interfere) with the laser.
Although elongation of the pores may increase the intensity along a preferred axis, a method to further induce emission along said axis will need to be found so as to increase the intensity of the laser and reduce the shielding required.
\newline
\newline
In order to conserve momentum, each annihilation will result in two photons travelling in opposite directions. Either a way to use both should be found, or shielding will be required to prevent damage/injury that would result from the unwanted beam.
